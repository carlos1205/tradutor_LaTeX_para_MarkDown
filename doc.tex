\documentclass[portuguese,brazilian,12pt,final]{article}
\usepackage[brazil]{babel}      % Pacotes da lingua portuguesa
\usepackage[utf8]{inputenc}     % com os caracteres com ç e acentuação
                                % e hifenização em português


\usepackage{isomath}       % Pacote matemático
\usepackage{natbib}        % habilita novos comandos de citação
\usepackage{amsmath}       % Pacote matemático
\usepackage{graphicx}      % Para inserir figuras
\usepackage{indentfirst}   % Para identar o primeiro paragrafo da seção



% Definição de novos comandos:
\newcommand{\Dx}{\Delta x}
\newcommand{\Dy}{\Delta y}
\newcommand{\imax}{i_{\max}}
\newcommand{\jmax}{j_{\min}}





% Comandos de formatação da página
\setlength{\baselineskip}{16.0pt}
\setlength{\parskip}{3pt plus 2pt}
\setlength{\parindent}{20pt}
\setlength{\oddsidemargin}{0.5cm}
\setlength{\evensidemargin}{0.5cm}
\setlength{\marginparsep}{0.75cm}
\setlength{\marginparwidth}{2.5cm}
\setlength{\marginparpush}{1.0cm}
\setlength{\textwidth}{150mm}

\begin{document}

{para Iniciantes Exemplo} nova linha

\section{Introdução}

Para inserir um novo parágrafo, deixe uma ou mias linhas em branco.
Note que todas as formulas estão entre
que signifíca \textit{modo matemático inline}.

A fonte padrão é Computer Modern, que inclui as variações \textit{itálico},
\textbf{negrito}

\section{Listas}

Alguns exemplos de listas: A seguir uma lista numerada

\begin{enumerate}
\item Pão
\item Queijo
\item Presunto
\end{enumerate}

A seguir uma lista não numerada

\begin{itemize}
\item Maria
\item João
\end{itemize}

\section{Figuras}


A Figuratem sua largura igual a 70\% da largura do texto.

Figuras e tabelas flutuam no texto. O tenta posicioná-las de forma a otimizar o máximo de espaço.

Usando o comando label é possível referenciar a equação depois.

A equação expressa a energia no modelo Ising.

É possível definir novo comandos, veja no preâmbulo logo após os usepackages. Foram definidos os comandos Dx, Dy, imax e jmax:

\end{document} 